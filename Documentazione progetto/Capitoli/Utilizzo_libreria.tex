Andiamo qui a riassumere alcuni passi operativi che sono stati realizzati durante lo sviluppo ed il proseguimento del progetto.

\section{Installazione driver camera}
TODO - download e installazione driver

Nel caso in cui la camera sembri non funzionare, come specificato nel capitolo ~\ref{section:deamonCameraUeye}, è necessario avviare il daemon della camera.
Questo è possibile solo se esso non è gia stato attivato come nel caso, per esempio, di un plug-in a caldo.
Il comando da runnare da terminale, in qualsiasi \textit{posizione}, è il seguente:
\begin{lstlisting}
	- sudo /etc/init.d/ueyeusbdrc start 
	(il comando iniziale permetteva di scegliere tra usb e eth: la nostra camera e usb)!
\end{lstlisting}

\section{Avvio del codice}
Nella nostra esperienza ci è stato molto comodo andare ad utilizzare non il classico terminale messo a disposizione da Ubuntu, ma bensì ci siamo appoggiati all'utilizzo di \textit{terminator}, il quale permette una migliore gestione di terminali multipli.

Ecco una sequenza di comandi da inserire da terminale per poter andare ad eseguire il codice. 
\begin{lstlisting}

Scrivere su tutti terminali (e su tutti quelli che verranno aperti) i seguenti comandi per il \textit{bash} del progetto:
	- CMD_1: cd ArrowFinder/devel/; . setup.bash
	- CMD_2: cd ..

Terminale 1:
	- Caricare il bash da devel (CMD_1 / CMD_2)
	- CMD_3: roscore

Terminale 2:
	- Caricare il bash da devel (CMD_1 / CMD_2)
	- CMD_4: rosrun arrow_finder arrow_finder_node
	
Terminale 3:
	- Caricare il bash da devel (CMD_1 / CMD_2)
	- CMD_5: cd ~/ArrowFinder/src/ueye_cam/launch
	- CMD_6: roslaunch debug.launch 

Terminale 4:
	- CMD_7: rosrun topic_tools throttle messages /camera/image_raw 5.0 
\end{lstlisting}
Da evidenziare come, il comando eseguito sul terminale 4, risulta essere utile nel caso in cui sia necessario andare a temporizzare, in maniera automatica e gestita completamente da ROS, la lettura dell'immagine