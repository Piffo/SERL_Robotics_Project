\section{Modifica dell'altezza della camera}
Per effettuare modifiche all'altezza della camera è sufficiente modificare il parametro $h_{cam}$ della matrice:
\begin{equation}
	R_{traslazione} = \\
	\begin{pmatrix}
	1 & 0 & 0 & 0 \\
	0 & 1 & 0 & 0 \\
	0 & 0 & 1 & -h_{cam}\\
	0 & 0 & 0 & 1 \\
	\end{pmatrix}
\end{equation}
Nel caso di modifiche anche rispetto ad altri assi quali U e V sarà sufficiente modificare i valori dei primi due elementi della quarta colonna che rappresentano rispettivamente traslazioni lungo l'asse U e l'asse V.

\section{Modifica dell'inclinazione della camera}
Per effettuare eventuali modifiche alla inclinazione della camera, sempre ammesso che il piano Z si voglia uscente dal piano della camera, sarà allora sufficiente modificare il parametro $\alpha$, che appare nella matrice seguente:
\begin{equation}
	R_{rotazione} =
	\begin{pmatrix}
		cos(\dfrac{\pi}{2}+\alpha) & -sin(\dfrac{\pi}{2}+\alpha) & 0 & 0 \\
		sin(\dfrac{\pi}{2}+\alpha) & cos(\dfrac{\pi}{2}+\alpha)& 0 & 0 \\
		0 & 0 & 1 & 0 \\
		0 & 0 & 0 & 1 \\
	\end{pmatrix}
\end{equation}

\section{Modifica del terreno su cui si trova il robot}
Nel caso in cui il robot su cui è montata la camera dovvesse essere utilizzato in ambienti diversi da quelli di un laboratorio in cui sono assenti salite, discese o dislivelli allora sarà necessario modificare il piano su cui la freccia giace.

Per fare ciò sarà necessario ricalcolare il piano su cui si trova la freccia $ax+by+cz+d=0$ e ricalcolare il fattore di scala
\begin{equation}
	\begin{split}
	s = \dfrac{-d}{aX'+bY'+c}	
	\end{split}
\end{equation}

Si nota come, se il robot dovesse viaggiare su un pavimento senza variazioni lungo l'asse U, allora il fattore di scala sarebbe definito come:
$$
s = \abs{\dfrac{h_{cam}
	}{-sin(\dfrac{\pi}{2}+\alpha)*Y'+ cos(\dfrac{\pi}{2}+\alpha)}}
$$